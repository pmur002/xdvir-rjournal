\setmainfont{Lato Light}
\fontsize{12pt}{17pt}\selectfont
\setlength{\columnsep}{1cm}
\begin{minipage}[t]{16.25cm}
\begin{multicols}{2} 
From a general and economic policy perspective, the entire period from
1820 to 1930 can be described as a relatively liberal period. With the
"Pacific War", as a result of which the nitrate mines were awarded to
Chile, the economy experienced a profound upswing. The period from
1940 to 1973 is generally seen as a phase in which the government
increasingly intervened in the economy and Chile was isolated
internationally. During the Allende regime (1971 to 1973), this policy
was exaggerated and the economy practically became a central
economy. The military regime (1973 to 1990) - despite numerous
violations of human rights - ensured liberalization of trade and
finance.
\end{multicols}
\end{minipage} 
